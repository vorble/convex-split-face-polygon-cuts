\documentclass[12pt]{article}
\usepackage{amsmath}
\begin{document}

\section*{Convex Split-Face Polygon Cuts}
{\em{January 8, 2022}}

\subsection*{Tools}

For more compact, finite summations:
\begin{equation}
    S(x_1,...,x_N) = \sum_{k=1}^{N} x_k
\end{equation}

\subsection*{Introduction}

There is a formula which counts the number of ways you can cut a convex polygon
between all its faces. Given a convex polygon with $N$ faces, this is
\begin{equation}
    C = S(N-1,...,0) = \frac{N (N - 1)}{2}
\end{equation}
This captures the idea that from one face, you can cut to $N-1$ other faces and from
the next you can cut to $N-2$ other faces while being sure not to double count.
\\
\\
What if the polygon's faces were further divided into some number of colinear
segments? How can we count the number of ways you can cut it?

\subsection*{Equations}

Let there be a convex polygon with $N$ faces. Each face numbered by $k$ with
$1 \le k \le N$ and each divided into $x_k$ segments.
Let $C(x_1,...,x_N)$ be the number of
cuts. For the simple case where each $x$ is equal to 1, it's the familiar
formula
\begin{equation}
    C(1,...,1) = S(N-1,...,0) = \frac{N (N - 1)}{2}
\end{equation}
In the general case, the count function satisfies a couple relations. First,
since the polygon in question has no set orientation and may be rotated, the
arguments to the count function may also be rotated
\begin{equation}
    C(x_1,...,x_k) = C(x_2,...,x_k,x_1)
\end{equation}
Second, extending the cut process described for the simple case in the introduction,
given a known count, dividing a face produces new cuts from each other faces'
segments
\begin{equation}
    C((x_1 + 1),...,x_k) = S(x_2,...,x_N) + C(x_1,...,x_N)
\end{equation}
This can be applied in a reductive manner by subtracting 1 from $x_1$
and repeatedly applying the rule until the first parameter to the
remaining application of $C$ is reduced to 1
\begin{equation}
\begin{aligned}
    C(x_1,...,x_k) &=           S(x_2,...,x_N) + C((x_1 - 1),...,x_N) \\
                   &=         2 S(x_2,...,x_N) + C((x_1 - 2),...,x_N) \\
                   &= (x_1 - 1) S(x_2,...,x_N) + C(1,x_2,...,x_N)
\end{aligned}
\end{equation}
Using the two rules together, each parameter to $C$ can be reduced to 1
and a fully reduced form can be derived
\begin{equation}
\begin{aligned}
    C(x_1,...,x_k) &= (x_1 - 1) S(x_2,...,x_N) \\
                   &+ (x_2 - 1) S(1,x_3,...,x_N) \\
                   &+ C(1,1,x_3,...,x_N) \\
                   &= (x_1 - 1) S(x_2,...,x_N) \\
                   &+ (x_2 - 1) S(1,x_3,...,x_N) \\
                   &+ (x_3 - 1) S(1,1,x_4,...,x_N) \\
                   &+ C(1,1,1,x_4,...,x_N) \\
                   &= \sum_{k=1}^{N} (x_k - 1) S(1,...,1,x_{k+1},...x_n) + C(1,...,1) \\
                   &= \sum_{k=1}^{N} (x_k - 1) \left[ (k-1) + S(x_{k+1},...x_n) \right] + C(1,...,1) \\
\end{aligned}
\end{equation}
This derivation process is useful when calculating these counts on paper.
The abbreviations can be expanded for the final result
\begin{equation}
\begin{aligned}
    C(x_1,...,x_k) &= \sum_{k=1}^{N} (x_k - 1) \left[ (k-1) + \sum_{j=k+1}^{N} x_j \right] + \frac{N(N-1)}{2}
\end{aligned}
\end{equation}
Computationally, the nested summations do not necessarily mean the calculation
will take $O(N^2)$ steps since the inner summation can be pre-calculated. Then,
for each $k$, reduce the calculated value by $x_k$.

\end{document}
